\documentclass[12pt]{article}

% PACKAGES
\usepackage[a4paper,bindingoffset=0in,
            left=1in,right=1in,top=1in,bottom=1.5in,
            footskip=.5in]{geometry}
\usepackage{lastpage}
\usepackage[ddmmyyyy]{datetime}
\usepackage{amsmath}              % need for subequations
\usepackage{graphicx}             % need for figures
\usepackage{verbatim}             % useful for program listings
\usepackage{color}                % use if color is used in text
\usepackage{subfigure}            % use for side-by-side figures
\usepackage{hyperref}             % use for hypertext links, including those to external documents and URLs
\usepackage{fancyhdr}             % for image in header    
\usepackage[table,xcdraw]{xcolor} % for color in cell
\usepackage{float}
\restylefloat{table}

\setlength{\parindent}{0pt}
\hypersetup{
    colorlinks=false,
    pdfborder={0 0 0},
}

%%%%%%%%%%%%%%%%%%%%%%%%%%%%%%%%%%%%%%%%%%%%%%%%%%%%%%%%%%%%%%%%%%%%%%%%%%%%%%%%%%%%%%%%%%%%%%%%%%%%%%%%%%

% HEADER and FOOTER
\pagestyle{fancy}{
  \lhead{\includegraphics[width=4.5cm]{Documents/Images/logoUnimi}}
  \rhead{\includegraphics[width=5cm]{Documents/Images/logoPong}}
  \lfoot{Last updated: \today}
  \cfoot{ }
  \rfoot{Page \thepage\ of \pageref{LastPage}}
}

\begin{document}

% logoTeam
\begin{center}
  \begin{figure}[H]
  \centering
  \vspace*{5\baselineskip}
  \includegraphics[width=10cm]{Documents/Images/logoTeam}
  \end{figure}

  \vspace{50pt}
  {\huge \textbf{Game Title}} \\
  {\large \textbf{ \textit{PONG - Game and Level Design}}}
\end{center}

\vspace{20pt}
\begin{table}[H]
  \centering
  \begin{tabular}{lcr}
    \textbf{Francesco Periti}	& \underline{\href{mailto:francesco.periti@studenti.unimi.it}{francesco.periti@studenti.unimi.it}}	& 930650 \\
    \textbf{Francesco Principe}	& \underline{\href{mailto:francesco.principe@studenti.unimi.it}{francesco.principe@studenti.unimi.it}}	& 937622 \\
    \textbf{Davide Valentini}	& \underline{\href{mailto:davide.valentini@studenti.unimi.it}{davide.valentini@studenti.unimi.it}}	& XXX \\
    \textbf{Elena Coperchini}	& \underline{\href{mailto:elena.coperchini@gmail.com}{elena.coperchini@gmail.com}}			& \\
  \end{tabular}
\end{table}


  \vspace{10pt}
% TABELLA 1
\begin{table}[H]
  \centering
  \begin{tabular}{|l|l|}
    \hline
    \cellcolor{lightgray}\textbf{Purpose} &  Define rules for files and directories \\\hline
    \cellcolor{lightgray}\textbf{Creation date} & 02/11/2018 \\\hline
    \cellcolor{lightgray}\textbf{Current owner} & Francesco Periti \\\hline
    \cellcolor{lightgray}\textbf{Last update} & \today \\\hline
  \end{tabular}
\end{table}

\clearpage

\section*{Revision History}
% TABELLA 2
\begin{table}[H]
\centering
\begin{tabular}{|l|l|l|}
\hline
\cellcolor{lightgray}\textbf{Who} & \cellcolor{lightgray}\textbf{When} & \cellcolor{lightgray}\textbf{What} \\ \hline
Francesco Periti & 02/11/2018 & Created this document \\ \hline
Francesco Periti & 03/11/2018 & Added some conventions to directory structure \\ \hline
Davide Valentini & 03/11/2018 & Added e-mail references \\ \hline
Francesco Principe & 04/11/2018 & Added some conventions to file naming \\ \hline
Davide Valentini & 04/11/2018 & Added software list \\ \hline
Francesco Principe & 05/11/2018 & Global revision \\ \hline
Davide Valentini & 06/11/2018 & Added file access rules \\ \hline
Francesco Periti & 06/11/2018 & Added repository structure rules \\ \hline
\end{tabular}
\end{table}

\clearpage

\section{Software List}

\subsection{Asset Editing Software}
\begin{itemize}
	\item \textbf{Photoshop CC 2018}: 2D bitmap images
	\item \textbf{Inkscape 0.92+}: SVG images
	%\item \textbf{Maya}: 3D assets
\end{itemize}

\subsection{Development Software}
Neverwinter Nights Toolset, GIT

\subsection{Organization Software}
TexMaker with MiKTeX 2.9

\subsection{Environments}
Windows 10

\section{Data Types and Format}

\subsection{Date format}
DD/MM/YYYY

\subsection{Text}
tex (LaTeX).

Use 4-spaces \textbf{tabs} for indentations.

Use \textbf{double return} to break paragraphs.

Use '\textbf{H}' to place floats (e.g. images and tables) in their correct position. Example: \textit{\textbackslash{}begin\{figure\}[H]}

Every LaTeX files has 100\% compatibility with any other LaTeX file.

\subsection{Images}
\textbf{Formats available for assets}: tif/tiff, png, svg.

\textbf{Formats available within documentation}: png.

\textbf{Formats available for reference images}: assets-available formats, jpeg, jpg, bmp, gif.

Size (W x H) is defined according to image category (texture, portrait, input button…).
\begin{itemize}
	\item \textbf{\path{./Documents/Images/Characters}}: 250px x 350px, Photoshop
	\item \textbf{\path{./Documents/Images/Circumplexes}}: 1321px x 895px, exported from \path{./Documents/Images/SVG/circumplexes.svg}
	\item \textbf{\path{./Documents/Images/Evolutions}}: 1321px x 895px, exported from \path{./Documents/Images/SVG/evolutions.svg}
	\item \textbf{\path{./Documents/Images/SVG}}: Inkscape
\end{itemize}

\subsection{Video}
To be defined.

Contact one of the project owners:
\begin{itemize}
	\item francesco.periti@studenti.unimi.it
	\item francesco.principe@studenti.unimi.it
	\item davide.valentini@studenti.unimi.it
\end{itemize}
mkv

\subsection{Audio}
Audacity
To be defined.

Contact one of the project owners:
\begin{itemize}
	\item francesco.periti@studenti.unimi.it
	\item francesco.principe@studenti.unimi.it
	\item davide.valentini@studenti.unimi.it
\end{itemize}
Sampling rate:

\subsection{3D Models}
To be defined.

Contact one of the project owners:
\begin{itemize}
	\item francesco.periti@studenti.unimi.it
	\item francesco.principe@studenti.unimi.it
	\item davide.valentini@studenti.unimi.it
\end{itemize}
Maximum number of triangles:

Scale: 

\subsection{etc etc}


\section{Data Storage and Access}
Project data are stored on a private git repository.

The repository structure follows the good pratices of GitFlow.

The history graph can become very complex; in order to avoid that, you have to work with two type of branch:
\begin{itemize}
\item branch with infinite lifetime;
\item branch with finite lifetime;
\end{itemize}

\subsection{Branch with infinite lifetime}
They are the main branch of the project. There are two different branches:
\begin{itemize}
\item \textbf{master}:\\
  this branch contains %the deployed project version%
  the last project review locked down by the project responsables.
    
  Only with the last version on master it can be done \textit{the level design to art handoff}.
  
  You have not to do any commit/merge in this branch unless the project responsables agree that the current version can be \textit{lock-down}.
\item \textbf{develop}:\\
  This branch is the development branch. Each have to work and make own changes to it.
  
  When you think you had made a good change, you have to do a commit it inside here so other can use it.
\end{itemize}

\subsection{Branch with finite lifetime}
You work on them just for a certain period and then you have to delete them.

You have to work on one at a time.

When you want to add a new content you have not to work on develop branch.

You have to create a your temporary branch where you can work and only when you have finished it
you have to merge your branch with the \textit{develop} branch.

You will never use more this branch so delete it. When you will want to add new content you have to make a new branch e so on. 

\subsection*{Synthesis}
When you work on a new content, you have to work on temporary branch and when you finish you have to merge your commit with the develop branch.\\
\textit{New content? New branch}.\\
\textit{Work completed? New merge with develop}.\\
And so on.


Only \textbf{editors} are allowed to add new files and edit existent ones. You can find the list of allowed editors for a folder in the file \textbf{!AdminAndEditors.txt} in the corresponding folder. If absent, refer to the same file in parent folder.

Only \textbf{admins} are allowed to add new editors. You can find the list of admins for a folder in the file \textbf{!AdminAndEditors.txt} in the corresponding folder. If absent, refer to the same file in parent folder.

\subsection{Backup}
There are daily backups on three different machines in three different locations.
Responsibles:
\begin{itemize}
	\item \textbf{Francesco Periti} for Piacenza
	\item \textbf{Francesco Principe} for Parma
	\item \textbf{Davide Valentini} for Milan
\end{itemize}

\section{Directory Structure}
\begin{itemize}
\item \textbf{\path{./Documents}}: contains all files used for the current level design document. Each subfolder but \textit{Images} matches a chapter of the Level Design Document.
  \begin{itemize}
    \item \textbf{\path{./Documents/levelDesignDocument.tex}}: this is the current level-design-document file. It contains references to the LaTeX files that compose it.

    \item \textbf{\path{./Documents/Characters}}: this directory contains all the LaTeX files about the characters.

    \item \textbf{\path{./Documents/Images}}: this directory contains every useful image for the documentation.

    \item \textbf{\path{./Documents/Story}}: this directory contains all the LaTeX files about the story.
  \end{itemize}
  \item \textbf{\path{./dataManagmentDocument.tex}}: this is the current data-managment-document file. It contains all technical features and rules to work on the project.
\end{itemize}

To add notes or comments to a file, create a txt file with the same name in the same folder.

\section{File Naming Convention}
During the project development has to be respected the following conventions:
\begin{itemize}
  \item The name of each directory and each file has to be significant and homogeneous.
  \item The name of each file has to begin with a lowercase character. Each time you want to separate two words within a file name, you have not to add a space between them but you have to make the first character of the second word uppercase: \textit{fileNameExample.ext}
  \item The name of each file has to begin with a uppercase character. Each time you want to separate two words within a directory name, you have not to add a space between them but you have to make the first character of the second word uppercase: \textit{FolderNameExample}
   \item Files named \textit{index.tex} are used as containers, they connect all the other files inside the current directory
\end{itemize}


\end{document}
