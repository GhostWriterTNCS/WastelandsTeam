\documentclass[12pt]{article}

% PACKAGES
\usepackage[a4paper,bindingoffset=0in,
            left=1in,right=1in,top=1in,bottom=1.5in,
            footskip=.5in]{geometry}
\usepackage{lastpage}
\usepackage[ddmmyyyy]{datetime}
\usepackage{amsmath}              % need for subequations
\usepackage{graphicx}             % need for figures
\usepackage{verbatim}             % useful for program listings
\usepackage{color}                % use if color is used in text
\usepackage{subfigure}            % use for side-by-side figures
\usepackage{hyperref}             % use for hypertext links, including those to external documents and URLs
\usepackage{fancyhdr}             % for image in header    
\usepackage[table,xcdraw]{xcolor} % for color in cell
\usepackage{float}
\restylefloat{table}

\setlength{\parindent}{0pt}
\hypersetup{
    colorlinks=false,
    pdfborder={0 0 0},
}

%%%%%%%%%%%%%%%%%%%%%%%%%%%%%%%%%%%%%%%%%%%%%%%%%%%%%%%%%%%%%%%%%%%%%%%%%%%%%%%%%%%%%%%%%%%%%%%%%%%%%%%%%%

% HEADER and FOOTER
\pagestyle{fancy}{
  \lhead{\includegraphics[width=4.5cm]{Documents/Images/logoUnimi}}
  \rhead{\includegraphics[width=5cm]{Documents/Images/logoPong}}
  \lfoot{Last updated: \today}
  \cfoot{ }
  \rfoot{Page \thepage\ of \pageref{LastPage}}
}

\begin{document}

% logoTeam
\begin{center}
  \begin{figure}[H]
  \centering
  \vspace*{5\baselineskip}
  \includegraphics[width=10cm]{Documents/Images/logoTeam}
  \end{figure}

  \vspace{50pt}
  {\huge \textbf{Game Title}} \\
  {\large \textbf{ \textit{PONG - Game and Level Design}}}
\end{center}

\vspace{20pt}
\begin{table}[H]
  \centering
  \begin{tabular}{lcr}
    \textbf{Francesco Periti}	& \underline{\href{mailto:francesco.periti@studenti.unimi.it}{francesco.periti@studenti.unimi.it}}	& 930650 \\
    \textbf{Francesco Principe}	& \underline{\href{mailto:francesco.principe@studenti.unimi.it}{francesco.principe@studenti.unimi.it}}	& 937622 \\
    \textbf{Davide Valentini}	& \underline{\href{mailto:davide.valentini@studenti.unimi.it}{davide.valentini@studenti.unimi.it}}	& XXX \\
    \textbf{Elena Coperchini}	& \underline{\href{mailto:elena.coperchini@gmail.com}{elena.coperchini@gmail.com}}			& \\
  \end{tabular}
\end{table}


  \vspace{10pt}
% TABELLA 1
\begin{table}[H]
  \centering
  \begin{tabular}{|l|l|}
    \hline
    \cellcolor{lightgray}\textbf{Purpose} &  Define rules for files and directories \\\hline
    \cellcolor{lightgray}\textbf{Creation date} & 02/11/2018 \\\hline
    \cellcolor{lightgray}\textbf{Current owner} & Francesco Periti \\\hline
    \cellcolor{lightgray}\textbf{Last modification} & \today \\\hline
  \end{tabular}
\end{table}

\clearpage

\section*{Revision History}
% TABELLA 2
\begin{table}[H]
\centering
\begin{tabular}{|l|l|l|}
\hline
\cellcolor{lightgray}\textbf{Who} & \cellcolor{lightgray}\textbf{When} & \cellcolor{lightgray}\textbf{What} \\ \hline
Francesco Periti & 02/11/2018 & Created this document \\ \hline
Francesco Periti & 03/11/2018 & Added some conventions to directory structure \\ \hline
Davide Valentini & 03/11/2018 & Added e-mail references \\ \hline
Francesco Principe & 04/11/2018 & Added some conventions to file naming \\ \hline
Davide Valentini & 04/11/2018 & Added software list \\ \hline
Francesco Principe & 05/11/2018 & Global revision \\ \hline
\end{tabular}
\end{table}

\clearpage

\section{Software List}
%Put here a list of the software you are going to use. Versions are required and do not underestimate OSs version.
%Once this list is public every member is committed to use it.

\subsection{Asset Editing Software}
\begin{itemize}
	\item \textbf{Photoshop CC 2018}: 2D bitmap images
	\item \textbf{Inkscape 0.92+}: SVG images
	%\item \textbf{Maya}: 3D assets
\end{itemize}

\subsection{Development Software}
Neverwinter Nights Toolset, GIT

\subsection{Organization Software}
TexMaker with MiKTeX 2.9

\subsection{Environments}
Windows 10

\section{Data Types and Format}
%This is very dependent from the previous section. Based on the tool you use there are preferred format. It is also true the opposite: can start from here to decide the software list.
%You should have one subsection for every category of data you need to manage.

\subsection{Date format}
DD/MM/YYYY

\subsection{Text}
In order to have elegant and tidy files, you have to work with \textbf{LaTeX} files.

Use 4-spaces \textbf{tabs} for indentations.

Use \textbf{double return} to break paragraphs.

Use '\textbf{H}' to place floats (e.g. images and tables) in their correct position. Example: \textit{\textbackslash{}begin\{figure\}[H]}

Every LaTeX files has 100\% compatibility with any other LaTeX file.

\subsection{Images}
\textbf{Formats available for assets}: tif/tiff, png, svg.

\textbf{Formats available within documentation}: png.

\textbf{Formats available for reference images}: assets-available formats, jpeg, jpg, bmp, gif.

Size (W x H) is defined according to image category (texture, portrait, input button…).
\begin{itemize}
	\item \textbf{\path{./Documents/Images/Characters}}: 250px x 350px, Photoshop
	\item \textbf{\path{./Documents/Images/Circumplexes}}: 1321px x 895px, exported from \path{./Documents/Images/SVG/circumplexes.svg}
	\item \textbf{\path{./Documents/Images/Evolutions}}: 1321px x 895px, exported from \path{./Documents/Images/SVG/evolutions.svg}
	\item \textbf{\path{./Documents/Images/SVG}}: Inkscape
\end{itemize}

\subsection{Video}
%mkv

\subsection{Audio}
%Sampling rate:

\subsection{3D Models}
%Maximum number of triangles:

%Scale: 

%\subsection{etc etc}
%For each category, you must set a policy about encoding parameters.
%
%Encoding parameters depend on the data type. For text, stating “microsoft doc” can do (then you have to see the previous section to understand which version of office to use). For video, you must set resolution, encoder, and encapsulation format at least. About resolution, you are not strictly required to set just one, you may have multiple based on the purpose of the image/video.
%
%You can also set rules about binary size or content, it is all up to you.
%
%DO NOT waste your time setting policies for data types you are not using. You can add them later, when introduced in the project.

\section{Data Storage and Access}
%Where the data (shared among the team) is stored and who is in charge to manage it.
%
%Put here all the required information to access the data. A new team member should be able to start working just starting from this document.
Project data are stored on a private git repository.

\subsection{Backup}
%How you do it and Who is in charge to perform it.
There are daily backups on three different machines in three different locations.
Responsibles:
\begin{itemize}
	\item \textbf{Francesco Periti} for Piacenza
	\item \textbf{Francesco Principe} for Parma
	\item \textbf{Davide Valentini} for Milan
\end{itemize}

\section{Directory Structure}
\begin{itemize}
\item \textbf{\path{./Documents}}: contains all files used for the current level design document. Each subfolder but \textit{Images} matches a chapter of the Level Design Document.
  \begin{itemize}
    \item \textbf{\path{./Documents/levelDesignDocument.tex}}: this is the current level-design-document file. It contains references to the LaTeX files that compose it.

    \item \textbf{\path{./Documents/Characters}}: this directory contains all the LaTeX files about the characters.

    \item \textbf{\path{./Documents/Images}}: this directory contains every useful image for the documentation.

    \item \textbf{\path{./Documents/Story}}: this directory contains all the LaTeX files about the story.
  \end{itemize}
  \item \textbf{\path{./dataManagmentDocument.tex}}: this is the current data-managment-document file. It contains all technical features and rules to work on the project.
\end{itemize}

%Devise a directory tree (or rules to create a directory tree) in a way that each file can be located in only one location.
%
%This part is not trivial, at all.
%
%No tree is perfect (once again, it depends on the project). Better give also rules about how to walk the tree.
%
%Pictures here could also be useful.
%
%TIP: In every directory, subdirectories should be on the same conceptual level. i.e., putting “textures” and “maps” as possible choices may not be a good idea.

\section{File Naming Convention}
During the project development has to be respected the following conventions:
\begin{itemize}
  \item The name of each directory and each file has to be significant and homogeneous.
  \item The name of each file has to begin with a lowercase character. Each time you want to separate two words within a file name, you have not to add a space between them but you have to make the first character of the second word uppercase: \textit{fileNameExample.ext}
  \item The name of each file has to begin with a uppercase character. Each time you want to separate two words within a directory name, you have not to add a space between them but you have to make the first character of the second word uppercase: \textit{FolderNameExample}
   \item Files named \textit{index.tex} are used as containers, they connect all the other files inside the current directory
\end{itemize}

%Provide rules name every kind of file (see also Sec. 3).
%It is a good option to provide labels for encoding (or size in case of pictures) but also labels for their function inside the project (e.g., to distinguish tile textures from mesh skins).
%
%TIP. You can use keywords (providing a conversion table) to avoid extremely long names.

\end{document}
