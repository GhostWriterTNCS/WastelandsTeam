\section{Skills}
The previous table shows how the game skills are distributed throughout the game. You can find some detail in this section. These are the skill that the player can use while impersonating Sophie:
\begin{itemize}
\item \textbf{Magic Door}: Sophie uses the flying castle's door to rapidly move through the levels.
\item \textbf{Talk to objects}: Sophie uses this skill to have dialogues with objects. Just some objects can interact with her. They give her information, tips, rewards or other.
\item \textbf{Magic lantern}: Sophie uses the magic lantern to control Calcifer during a fight.
\item \textbf{Lie}: during a dialogue with someone Sophie chooses to lie to him.
\item \textbf{Persuasion}: during a dialogue with someone Sophie chooses to try to persuade him.
\item \textbf{Corruption}: during a dialogue with someone Sophie chooses to try to corrupt him.
\item \textbf{Lantern transformation}: during the game Calcifer can use scraps, objects or other to transform temporarily his lantern in some useful machine. %Sophie can control this skills. Just in some part of the game she has to use.
\item \textbf{Sewing}: Sophie can sew magic hats with the magic needle in the flying castle. Each hat requires some crafting materials.
\item \textbf{Flying}: Sophie pilots the flying castle.
\item \textbf{Sailing}: Sophie use this skill to sail a boat.
\item \textbf{Metamorphosis}: Sophie can allow Calcifer to turn into a powerful demon. This skill is used only in the last level where the player will control Calcifer to defeat Mizar.
\end{itemize}

\subsection{Magic lantern skills}
There are several combinations of attack that Calcifer can do in the game, they are not essential to finish the game or overpass levels. They just increase in some way the value attack:\\
TODO
\subsubsection{Pugno di fuoco}
\begin{table}[H]
  \centering
\begin{tabular}{|
>{\columncolor[HTML]{C0C0C0}}l |l|}
\hline
\textbf{Name} & Pugno di fuoco \\ \hline
\textbf{Saving Throw} &  \\ \hline
\textbf{Description} &  \\ \hline
\textbf{Range} &  \\ \hline
\textbf{Duration} &  \\ \hline
\textbf{Description} &  \\ \hline
\end{tabular}
\end{table}
\subsubsection{Calcio volante multiplo}
\begin{table}[H]
  \centering
\begin{tabular}{|
>{\columncolor[HTML]{C0C0C0}}l |l|}
\hline
\textbf{Name} & Pugno di fuoco \\ \hline
\textbf{Saving Throw} &  \\ \hline
\textbf{Description} &  \\ \hline
\textbf{Range} &  \\ \hline
\textbf{Duration} &  \\ \hline
\textbf{Description} &  \\ \hline
\end{tabular}
\end{table}
