\section{Skills}
The previous table shows how the game skills are distributed throughout the game. You can find some detail in this section. These are the skill that player can use while impersonate sophie:
\begin{itemize}
\item \textit{Magic Door}: Sophie uses the flying castle's door to move rapidly in to the world
\item \textit{Talk to objects}: Sophie use this skills to have dialogues with objects. Just some objects can interact with her. They give her information, tips, rewards or other.
\item \textit{Magic lantern}: Sophie uses the magic lantern to control Calcifer during a fight.
\item \textit{Lie}: During a dialogues with someone Sophie chooses to lie to him.
\item \textit{Persuasion}: During a dialogues with someone Sophie chooses to try to persuade him.
\item \textit{Corruption}: During a dialogues with someone Sophie chooses to try to corrupt him.
\item \textit{Lantern trasformation}: During the game Calcifer can use scraps, objects or other to trasform temporarily his lantern in soma usuful gadget. Sophie can control this skills. Just in some part of the game she have to use.
\item \textit{Sewing}: Sophie can sew magic hats with the magic needle in the flying castle. Each hat requires own crafting materials.
\item \textit{Flying}: Sophie pilots the flying castle.
\item \textit{Sailing}: Sophie use this skill to sail a boat.
\item \textit{Metamorphosis}: Sophie can allow Calcifer to turn into a powerful demon. This skill is used only in the last level where the player will control Calcifer to defeat Mizar.
\end{itemize}

\subsection{Magic lantern skills}
There are several combinations of attack that Calcifer can do in the game, they are not essential to finish the game or overpass levels. They just increase in some way the value attack:\\
TODO
\subsubsection{Pugno di fuoco}
\begin{table}[H]
  \centering
\begin{tabular}{|
>{\columncolor[HTML]{C0C0C0}}l |l|}
\hline
\textbf{Name} & Pugno di fuoco \\ \hline
\textbf{Saving Throw} &  \\ \hline
\textbf{Description} &  \\ \hline
\textbf{Range} &  \\ \hline
\textbf{Duration} &  \\ \hline
\textbf{Description} &  \\ \hline
\end{tabular}
\end{table}
\subsubsection{Calcio volante multiplo}
\begin{table}[H]
  \centering
\begin{tabular}{|
>{\columncolor[HTML]{C0C0C0}}l |l|}
\hline
\textbf{Name} & Pugno di fuoco \\ \hline
\textbf{Saving Throw} &  \\ \hline
\textbf{Description} &  \\ \hline
\textbf{Range} &  \\ \hline
\textbf{Duration} &  \\ \hline
\textbf{Description} &  \\ \hline
\end{tabular}
\end{table}
