\section{Skills}
During the game the player can use many skills while impersonating Sophie.

The following table shows how the skills are distributed throughout the game.

\begin{longtable}[H]{|p{2cm}|p{1.5cm}|p{1.7cm}|p{1.7cm}|p{1.7cm}|p{1.7cm}|p{1.5cm}|p{1.5cm}|}
  \hline
\cellcolor[HTML]{656565}{\color[HTML]{FFFFFF} \textbf{Skill}} & \cellcolor[HTML]{C0C0C0}{\color[HTML]{330001} \textbf{First steps}} & \cellcolor[HTML]{C0C0C0}{\color[HTML]{330001} \textbf{Where is Howl?}} & \cellcolor[HTML]{C0C0C0}{\color[HTML]{330001} \textbf{In enemy territory}} & \cellcolor[HTML]{C0C0C0}{\color[HTML]{330001} \textbf{Nasty surprise(s)}} & \cellcolor[HTML]{C0C0C0}{\color[HTML]{330001} \textbf{The djiin of the desert}} & \cellcolor[HTML]{C0C0C0}{\color[HTML]{330001} \textbf{The spirts realm}} & \cellcolor[HTML]{C0C0C0}{\color[HTML]{330001} \textbf{Fire and secrets}} \\ \hline
\textbf{Magic Door} & X & X & X & X &  & X &  \\ \hline
\textbf{Sewing} & X & O & O & O & O & O & O \\ \hline
\textbf{Talk to objects} & X & X & X & O & O & X & X \\ \hline
%\textbf{Magic lantern} &  & O & X & X &  & X & X \\ \hline
\textbf{Multi-option dialogue} &  & X & X & X & X & X &  \\ \hline
%\textbf{Lantern trasformation} &  &  & X &  & X & X &  \\ \hline
\textbf{Pilot the castle} &  &  &  &  & X &  &X  \\ \hline
%\textbf{Sailing} &  &  & X &  &  & X &  \\ \hline
\textbf{Metamor-phosis} &  &  &  &  &  &  & X \\ \hline
\end{longtable}

\textbf{X}: the skill is required \\
\textbf{O}: the skill may be used, but it is not required

\begin{itemize}
\item \textbf{Magic door}: Sophie uses the flying castle's magic door to rapidly move through the levels.
\item \textbf{Sewing}: Sophie can sew magic hats with the magic needle in the flying castle. Each hat requires some crafting materials.
\item \textbf{Talk to objects}: Sophie uses this skill to have dialogues with objects. Just some objects can interact with her. They give her information, tips, rewards or other.
%\item \textbf{Magic lantern}: Sophie uses the magic lantern to control Calcifer during a fight.
\item \textbf{Multi-option dialogue}: during a dialogue Sophie has some options to choose from.
%\item \textbf{Lantern transformation}: during the game Calcifer can use scraps, objects or other to transform temporarily his lantern in some useful machine. %Sophie can control this skills. Just in some part of the game she has to use.
\item \textbf{Pilot the castle}: Sophie pilots the flying castle.
%\item \textbf{Sailing}: Sophie use this skill to sail a boat.
\item \textbf{Metamorphosis}: Sophie can allow Calcifer to turn into a powerful demon. This skill is used only in the last level where the player will control Calcifer to defeat Mizar.
\end{itemize}

\subsection{Combat skills}
In order to defeat the enemis, Sophie can use a magic lantern to control Calcifer during a fight.

Where not specified, the damage inflicted by an attack corresponds to the one established by the current magic lantern.

\subsubsection{Fire fist}
\textbf{Unlocked at level}: 2 \\
\textbf{Range (m)}: 1.5 (melee) \\
\textbf{Saving Throw}: no Saving Throw

A fire fist that hits the enemy in front of Sophie.

\subsubsection{Fire bolt}
\textbf{Unlocked at level}: 3 \\
\textbf{Range (m)}: 10, 20, 30 \\
\textbf{Saving Throw}: no Saving Throw

A fire bolt that hits an enemy in its range.

\subsubsection{Fire wave}
\textbf{Unlocked at level}: 5 \\
\textbf{Range (m)}: 5, 10, 15 \\
\textbf{Saving Throw}: no Saving Throw

A fire wave that hits all the enemies in its range.

%There are several combinations of attack that Calcifer can do in the game, they are not essential to finish the game or overpass levels. They just increase in some way the value attack:\\
%\subsubsection{Pugno di fuoco}
%\begin{table}[H]
%  \centering
%\begin{tabular}{|
%>{\columncolor[HTML]{C0C0C0}}l |l|}
%\hline
%\textbf{Name} & Pugno di fuoco \\ \hline
%\textbf{Saving Throw} &  \\ \hline
%\textbf{Description} &  \\ \hline
%\textbf{Range} &  \\ \hline
%\textbf{Duration} &  \\ \hline
%\textbf{Description} &  \\ \hline
%\end{tabular}
%\end{table}
%\subsubsection{Calcio volante multiplo}
%\begin{table}[H]
%  \centering
%\begin{tabular}{|
%>{\columncolor[HTML]{C0C0C0}}l |l|}
%\hline
%\textbf{Name} & Pugno di fuoco \\ \hline
%\textbf{Saving Throw} &  \\ \hline
%\textbf{Description} &  \\ \hline
%\textbf{Range} &  \\ \hline
%\textbf{Duration} &  \\ \hline
%\textbf{Description} &  \\ \hline
%\end{tabular}
%\end{table}
