\section{Data Storage and Access}
Project data are stored on a private git repository:\\
	\textit{git@pong.di.unimi.it:GLD/18/teamdellelande}

The repository structure follows the good pratices of GitFlow.

The history graph can become very complex; in order to avoid that, you have to work with two type of branches:
\begin{itemize}
	\item branches with infinite lifetime
	\item branches with finite lifetime
\end{itemize}

\subsection{Branches with infinite lifetime}
They are the main branches of the project. There are two different infinite lifetime branches:
\begin{itemize}
\item \textbf{master}: this branch contains the last project review locked down by the project responsible.

  Only with the last version on master it can be done the \textit{level design to art handoff}.

  You have not to do any commit/merge in this branch unless the project responsibles agree that the current version can be \textit{locked-down}.

\item \textbf{develop}: This branch is the development branch. Each one has to work and make his own changes to it.

  When you think you had made a good change, you have to do a commit inside here so others can use it.
  If the project responsibles agree that the current version can be \textit{locked-down} the develop will be merged on the \textit{master} branch and a TAG version will be binded to that commit.
\end{itemize}

\subsection{Branches with finite lifetime}
You work on them just for a certain period and then you have to delete them.

You have to work on one branch at a time.

When you want to add a new content you have not to work on \textit{develop} branch.

You have to create a temporary branch where you can work and only when you have finished it you have to merge your branch with the \textit{develop} branch.

%You will never use more this branch so delete it. When you will want to add new content you have to make a new branch e so on.

\subsection*{Synthesis}
When you work on a new content, you have to work on a temporary branch and when you have finished you have to merge your commits with the \textit{develop} branch.\\
\textit{New content? New branch}.\\
\textit{Work completed? Merge with develop}.\\
%And so on.

\subsection*{Revision history}
Whenever you make a commit or a merge with the \textit{develop} branch, you must add a comment about your work in the revision history file: \path{./Documents/revisionHistory.tex}.

\subsection{Data access and permissions}
The following table indicates \textit{owners} and \textit{editors} for files.

Only the \textit{editors} are allowed to add and edit files.

Only the \textit{owners} are allowed to enable new \textit{editors}, edit the folder and add subfolders. The \textit{owners} are not allowed to edit files (unless they are also \textit{editors}).

To avoid merging non-text files, nobody has to add/edit non-text files in any directory, unless he is the owner of the directory and that is a subdirectory of the Images directory.

Everybody can read all the files.

Rules are sorted by priority: first rule has the highest priority and overwrites the others.

\begin{table}[H]
\centering
  \begin{tabularx}{\textwidth}{|X|p{3.5cm}|X|}
\hline
\cellcolor{lightgray}\textbf{Files} & \cellcolor{lightgray}\textbf{Owner} & \cellcolor{lightgray}\textbf{Editors} \\ \hline
\path{./Documents/LevelDesignDocument/Images/Characters/*} & Francesco Principe & Francesco Principe \\ \hline
\path{./Documents/LevelDesignDocument/Images/Enemies/*} & Francesco Principe & Francesco Principe \\ \hline
\path{./Documents/LevelDesignDocument/Images/Location/*} & Francesco Principe & Francesco Principe \\ \hline
\path{./Documents/LevelDesignDocument/Images/Landmarks/*} & Francesco Principe & Francesco Principe \\ \hline
\path{./Documents/LevelDesignDocument/Images/Maps/*} & Davide Valentini & Davide Valentini \\ \hline
\path{./Documents/LevelDesignDocument/Images/Puzzles/*} & Davide Valentini & Davide Valentini \\ \hline
\path{./Documents/LevelDesignDocument/Images/Clothes/*} & Francesco Periti & Francesco  Principe,  Francesco
Periti, Davide Valentini \\ \hline
\path{./Documents/LevelDesignDocument/Images/Hats/*} & Francesco Periti & Francesco  Principe,  Francesco
Periti, Davide Valentini \\ \hline
\path{./Documents/LevelDesignDocument/Images/Lanterns/*} & Francesco Periti & Francesco  Principe,  Francesco
Periti, Davide Valentini \\ \hline
\path{./Documents/LevelDesignDocument/Images/CraftingMaterials/*} & Francesco Periti & Francesco  Principe,  Francesco
Periti, Davide Valentini \\ \hline
\path{./Documents/LevelDesignDocument/Images/Diagrams/*} & Davide Valentini & Davide Valentini \\ \hline
\path{./Documents/LevelDesignDocument/Images/*} & Francesco Periti & Francesco Principe, Francesco Periti, Davide Valentini \\ \hline
\path{./Documents/DataManagmentDocument/*} & Francesco Periti & Francesco Periti \\ \hline
\path{./Documents/*.tex} & Francesco Principe & Every member of the Wastelands Team \\ \hline
\path{./References/*} & Davide Valentini & Every member of the Wastelands Team can add files \\ \hline
\path{./Logos/*} & Francesco Periti & Francesco Periti \\ \hline
\path{./*} & Francesco Principe & Francesco Principe, Francesco Periti, Davide Valentini \\ \hline
\end{tabularx}
\end{table}

\subsection{Backup}
There are daily backups on three different machines in three different locations.

Responsibles:
\begin{itemize}
	\item \textbf{Davide Valentini} for Milan (master)
	\item \textbf{Francesco Periti} for Piacenza (slave)
	\item \textbf{Francesco Principe} for Parma (slave)
\end{itemize}
