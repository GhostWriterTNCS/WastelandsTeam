\subsection{General conventions}
During the project development has always to be respected the following generic conventions:
\begin{itemize}
  \item The name of each file has to begin with a lowercase character. Each time you want to separate two words within a file name, you have not to add a space between them but you have to make the first character of the second word uppercase: \textit{fileNameExample.ext}
  \item The name of each folder has to begin with an uppercase character. Each time you want to separate two words within a directory name, you have not to add a space between them but you have to make the first character of the second word uppercase: \textit{FolderNameExample}
  \item Files named \textit{index.tex} are used as containers, they connect all the other files inside the current directory. Each chapter directory has to contain one.
  \item Never separate words.
\end{itemize}
\subsection{Specific conventions}
\begin{longtable}[H]{|p{8cm}|p{8cm}|}
\hline
\rowcolor[HTML]{9B9B9B} 
\multicolumn{1}{|l|}{\cellcolor[HTML]{9B9B9B}\textbf{PATH}} & \multicolumn{1}{l|}{\cellcolor[HTML]{9B9B9B}\textbf{CONVENTIONS}}                 \\ \hline
\path{/Documents/} &
Each directory inside this, is bound with a document.

So, each document has a directory named after its own name.

If you want to add a document you have to add also the relative directory.

Example: \textit{DocumentName} \\\hline

\path{/Documents/LevelDesignDocument} &
Each directory inside this, expect the Images folder, is bound with a chapter.

So, each chapter has a directory named after its own name.

If you want to add a chapter you have to add also the relative directory.

Example: \textit{ChapterName}\\\hline

\path{/Documents/LevelDesignDocument/ChapterName} &
Inside a chapter directory, each file has to be named after the section of the chapter.

If you want add a section you have to add also the relative file \textit{.tex}.

Example: \textit{sectionName.tex} \\\hline

\path{/Documents/LevelDesignDocument/Images/Characters} &
The name of each file inside here has to be the name of the character.

Example: \textit{sophie.png} \\\hline

\path{/Documents/LevelDesignDocument/Images/Locations}  &
The name of each file inside here has to be the name of the location.

Example: \textit{castleOfKingsbury.png} \\\hline

\path{/Logos} &
The name of each image inside here has to be the concatenation of two strings:
  \begin{itemize}
   \item logo
   \item the name of the logo
   \end{itemize}
Example: \textit{logoUnimi.png} \\\hline

\path{/Documents/LevelDesignDocument/Images/Maps}       &
The name of each file inside here has to be the name of the location.

Example: \textit{kingsbury.png} \\

\path{/Documents/LevelDesignDocument/Images/Diagrams}   & Each file with extension \textit{.svg} has to have the corrispective file (with the corrispective name) in the exported directory with the

  extension \textit{.png}. Each file inside a subdirectory has to be named as the concatenation of two strings:
   \begin{itemize}
   \item character name
   \item directory name (singular)
   \end{itemize}
   Example: \textit{sophieEvolution.png}

   Each other file has to have the name of the content. \\\hline
   
   \path{/References/Image/} &
   Subfolders name are the same as the subfolder contained in \path{/Documents/LevelDesignDocument/Images/}.

   Each file inside a subdirectory has to be named as the concatenation of three strings:

\begin{itemize}
\item a number, the count of the images with the same name.
\item character name
\item directory name (singular)
\end{itemize}

Example: \textit{5dynamiaLocation.png}



Each other file has to be named after its content following the general conventions. \\\hline
\path{/Documents/DataManagementDocument/} &
   Each file inside this, is bound with a section.

So, each section has a file named after its own name.

If you want to add a section you have to add also the relative file. \\\hline

\end{longtable}
