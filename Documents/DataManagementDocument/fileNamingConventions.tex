\subsection{General conventions}
During the project development you have to respect the following generic conventions:
\begin{itemize}
  \item The name of each file has to begin with a lowercase character. Each time you want to separate two words within a file name, you have not to add a space between them but you have to make the first character of the second word uppercase. If you have more than one version of the file, add an underscore before the file type identifier: \textit{fileNameExample\_type.ext}
  \item The name of each folder has to begin with an uppercase character. Each time you want to separate two words within a directory name, you have not to add a space between them but you have to make the first character of the second word uppercase: \textit{FolderNameExample}
  \item Files named \textit{index.tex} are used as containers, they connect all the other files inside the current directory. Each chapter directory has to contain one.
  %\item Never separate words.
\end{itemize}

\newpage

\subsection{Specific conventions}
\begin{longtable}[H]{|p{8cm}|p{8cm}|}
\hline
\rowcolor[HTML]{9B9B9B} 
\multicolumn{1}{|l|}{\cellcolor[HTML]{9B9B9B}\textbf{PATH}} & \multicolumn{1}{l|}{\cellcolor[HTML]{9B9B9B}\textbf{CONVENTIONS}}                 \\ \hline
\path{/Documents/} &
Each directory inside this is bound with a document.

So, each document has a directory named after its own name.

If you want to add a document you have to add also the relative directory.

Example: \textit{DocumentName} \\\hline

\path{/Documents/LevelDesignDocument} &
Each directory inside this, except for the Images folder, is bound with a chapter.

So, each chapter has a directory named after its own name.

If you want to add a chapter you have to add also the relative directory.

Example: \textit{ChapterName}\\\hline

\path{/Documents/LevelDesignDocument/ChapterName} &
Inside a chapter directory, each file has to be named after the section of the chapter.

If you want add a section you have to add also the relative file \textit{.tex}.

Example: \textit{sectionName.tex} \\\hline

\path{/Documents/LevelDesignDocument/Images/Characters} &
The name of each file inside here has to be the name of the character. There is only one image for each character.

Example: \textit{sophie.png} \\\hline

\path{/Documents/LevelDesignDocument/Images/Enemies} &
The name of each file inside here has to be the type of enemy. There is only one image for each enemy.

Example: \textit{guard.png} \\\hline

\path{/Documents/LevelDesignDocument/Images/Locations}  &
The name of each file inside here has to be the name of the location. There is only one image for each location.

Example: \textit{castleOfKingsbury.png} \\\hline

\path{/Documents/LevelDesignDocument/Images/Landmarks}  &
The name of each file inside here has to be the name of the landmark. There is only one image for each landmark.

Example: \textit{deadEnd.png} \\\hline

\path{/Documents/LevelDesignDocument/Images/Maps}       &
The name of each file inside here has to be the name of the location represented by the map or the name of a landmark inside the map.

Example: \textit{dynamia.png} \\
Example: \textit{deadEnd.png} \\\hline

\path{/Documents/LevelDesignDocument/Images/Clothes}       &
The name of each file inside here has to be the name of a cloth collectable in the game. There is only one image for each cloth.

Example: \textit{guardOfDynamia.png} \\ \hline

\path{/Documents/LevelDesignDocument/Images/Hats}       &
The name of each file inside here has to be the name of a hat collectable in the game.  There is only one image for each hat.

Example: \textit{christmas.png} \\ \hline

\path{/Documents/LevelDesignDocument/Images/Lanterns}       &
The name of each file inside here has to be the name of a lantern collectable in the game. There is only one image for each lantern.

Example: \textit{demoniac.png} \\ \hline

\path{/Documents/LevelDesignDocument/Images/CraftingMaterials}       &
The name of each file inside here has to be the name of a crafting material in the game. There is only one image for each material.

Example: \textit{whiteWool.png} \\ \hline

\path{/Documents/LevelDesignDocument/Images/Puzzles}       &
The name of each file inside here has to be named as the concatenation of two strings through a low dash:
\begin{itemize}
\item map name
  \item counter starting from 1
\end{itemize}

Example: \textit{dynamia\_count1.png} \\ \hline

\path{/Documents/LevelDesignDocument/Images/Machines}       &
The name of each file inside here has to be named as the name of the Calcifer transformation that represents.

Example: \textit{horse.png} \\ \hline

\path{/Documents/LevelDesignDocument/Images/Palettes}       &
The name of each file inside here has to be named as the name of the locations that represent the colors inside it. Each file have to contain four colors.

Example: \textit{dynamia.png} \\ \hline

\path{/Documents/LevelDesignDocument/Images/Diagrams}   & For each \textit{.svg} file create a \textit{.png} file with the same name exported from the \textit{.svg} file itself.

	Each file inside a subdirectory has to be named as the concatenation of two strings:
   \begin{itemize}
   \item character name
   \item directory name (singular)
   \end{itemize}
   Example: \textit{sophieEvolution.png}

   Each other file has to have a name which clarify the content.
   
   Example: \textit{worldDiagram.svg} \\\hline
   
\path{/Documents/DataManagementDocument/} &
   Each file inside this, is bound with a section.

So, each section has a file named after its own name.

If you want to add a section you have to add also the relative file. \\\hline

\path{/Logos} &
The name of each image inside here has to be the concatenation of two strings:
  \begin{itemize}
   \item logo
   \item the name of the logo
   \end{itemize}
Example: \textit{logoUnimi\_small.png} \\\hline

\path{/References/Images/} &
   Folders inside this directory are named according to their topic. You can use up to three words in the name of the topic.

   %Each file inside a subdirectory has to be named as the concatenation of three strings:

\begin{itemize}
\item topic name
%\item directory name (singular)
\item underscore
\item counter starting from 1
\end{itemize}

Example: \textit{Dynamia/dynamia\_5.png} \\\hline

%Each other file has to be named after its content following the general conventions.

\end{longtable}
