\section{Website}
We have a website that we use as a container for reference images. \\Its url is \url{http://wastelandsteam.altervista.org/}.\\

There are two way to login: 
\begin{itemize}
\item \textit{guest mode}: the site is protected by password. Stakeholders can access to it using this password: \textit{gld18}.\\\\
  This mode allows to see images on the website but they can't edit them. The password is periodically updated by the admin and you can find the latest version in this document.
\item \textit{editor mode}: the site is protected by password. This mode gives you read and write permission on the site.\\\\
  If you want login you have to ask the admin for the credentials.
\end{itemize}

\subsection{Structure}
The structure of the site is built according to the \textit{Reference/Images} directory on the repository.
This means that the path of that directory loosely corresponds to the index of the site.\\\\
Example: \\
\textit{teamdellelande/References/Images} \\corresponds to\\ \url{http://wastelandsteam.altervista.org/}.\\\\
Example: \\
\textit{teamdellelande/References/Images/Dynamia} \\corresponds to\\ \url{http://wastelandsteam.altervista.org/dynamia}.\\\\
Each time you add an image in the site you have to add it also in the right directory on the repository and vice-versa.
The rules and permissions on the site (in editor mode) are the same of that directory with this little exception:
you can't use uppercase character in the urls. So each character is lowercase and to avoid spaces between words you have to insert a dash between them. Furthermore, to reference on the site to a subdirectory inside a directory on the repository, you can set the upper directory as parent post.
