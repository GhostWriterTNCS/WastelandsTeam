\documentclass[12pt]{article}

% PACKAGE - WHY?
\usepackage{amsmath}              % need for subequations
\usepackage{graphicx}             % need for figures
\usepackage{verbatim}             % useful for program listings
\usepackage{color}                % use if color is used in text
\usepackage{subfigure}            % use for side-by-side figures
\usepackage{hyperref}             % use for hypertext links, including those to external documents and URLs
\usepackage{fancyhdr}             % for image in header    
\usepackage[table,xcdraw]{xcolor} % for color in cell
%%%%%%%%%%%%%%%%%%%%%%%%%%%%%%%%%%%%%%%%%%%%%%%%%%%%%%%%%%%%%%%%%%%%%%%%%%%%%%%%%%%%%%%%%%%%%%%%%%%%%%%%%%

% IMMAGINI HEADER
\thispagestyle{fancy}{
\lhead{\includegraphics[width=4.5cm]{images/logoUnimi}}
\rhead{\includegraphics[width=6.5cm]{images/logoPong}}
}



\begin{document}

% logoTeam
\begin{center}
  \begin{figure}
    \centering
  \vspace*{5\baselineskip}
  \includegraphics[width=10cm]{images/logoTeam}
  \end{figure}

  {\large \textbf{Game Title}} \\
  \textbf{\textit{PONG - Game and Level Design}} \\
  {\small Updated 31 October 2018} \\

\begin{tabular}{lcr}\\\\
\textbf{Francesco Periti}    & Francesco.Periti@studenti.unimi.it   & 930650\\
\textbf{Francesco Principe}  & Francesco.Principe@studenti.unimi.it & 937622\\
\textbf{Davide Valentini}    & Davide.Valentini@studenti.unimi.it   & XXX\\
\textbf{Elena Coperchini}   & Elena.Coperchini@gmail.com &\\
\end{tabular}


% TABELLA 1
\begin{table}[]
  \begin{tabular}{|l|l||}
    \hline
    \cellcolor{gray}\textbf{Purpose} &  Define rules for files and directories \\\hline
    \cellcolor{gray}\textbf{Creation date} & 2/11/2018 \\\hline
    \cellcolor{gray}\textbf{Current owner} & Francesco Periti \\\hline
    \cellcolor{gray}\textbf{Last modification} & 2/11/2018\\   \hline
  \end{tabular}
\end{table}

% TABELLA 2
\begin{table}[]
\begin{tabular}{lll}
\hline
\multicolumn{3}{|l|}{\cellcolor{gray}\textbf{Revision History}}\\ \hline
\multicolumn{1}{|l|}{\cellcolor{gray}\textbf{Who}} & \multicolumn{1}{l|}{\cellcolor{gray}\textbf{When}} & \multicolumn{1}{l|}{\cellcolor{gray}\textbf{What}}\\ \hline
\multicolumn{1}{|l|}{Francesco Periti} & \multicolumn{1}{l|}{2/11/2018} & \multicolumn{1}{l|}{Created this document}\\ \hline
\multicolumn{1}{|l|}{Someone} & \multicolumn{1}{l|}{2/11/2018} & \multicolumn{1}{l|}{Added some content}\\ \hline
\multicolumn{1}{|l|}{Someone} & \multicolumn{1}{l|}{2/11/2018} & \multicolumn{1}{l|}{Created a reference section}\\ \hline
\end{tabular}
\end{table}

\end{center}

\section{Software List}
Put here a list of the software you are going to use. Versions are required and do not underestimate OSs version.
Once this list is public every member is committed to use it.
Whatever is not listed here can is supposed to be free for everyone to choose (recipe for disaster).
\subsection{Asset Editing Software}
Blender, Photoshop CC 2018, Maya
\subsection{Development Software}
NWN Toolset, GIT
\subsection{Organization Software}
Office 2016, LaTeX
\subsection{Environments}
Windows 10
\section{Data Types and Format}
This is very dependent from the previous section. Based on the tool you use there are preferred format. It is also true the opposite: can start from here to decide the software list.
You should have one subsection for every category of data you need to manage. 
\subsection{Text}
Microsoft Word 2016
Google Drive Document
\subsection{Images}
tif, png, svg
jpeg, jpg, bmp, gif for refence pictures
Size defined according to image category (texture, portrait, input button…)
\subsection{Video}
mkv
\subsection{Audio}
Sampling rate:
\subsection{3D Models}
Maximum number of triangles: 
Scale: 
\subsection{etc etc}

For each category, you must set a policy about encoding parameters.
Encoding parameters depend on the data type. For text, stating “microsoft doc” can do (then you have to see the previous section to understand which version of office to use). For video, you must set resolution, encoder, and encapsulation format at least. About resolution, you are not strictly required to set just one, you may have multiple based on the purpose of the image/video.
You can also set rules about binary size or content, it is all up to you.

DO NOT waste your time setting policies for data types you are not using. You can add them later, when introduced in the project.
\section{Data Storage and Access}
Where the data (shared among the team) is stored and who is in charge to manage it.
Put here all the required information to access the data. A new team member should be able to start working just starting from this document.
\subsection{Backup}
How you do it and Who is in charge to perform it.
\section{Directory Structure}
Devise a directory tree (or rules to create a directory tree) in a way that each file can be located in only one location.
This part is not trivial, at all.
No tree is perfect (once again, it depends on the project). Better give also rules about how to walk the tree.
Pictures here could also be useful.

TIP: In every directory, subdirectories should be on the same conceptual level. i.e., putting “textures” and “maps” as possible choices may not be a good idea.
\section{File Naming Convention}
Provide rules name every kind of file (see also Sec. 3).
It is a good option to provide labels for encoding (or size in case of pictures) but also labels for their function inside the project (e.g., to distinguish tile textures from mesh skins).

TIP. You can use keywords (providing a conversion table) to avoid extremely long names.





\end{document}
